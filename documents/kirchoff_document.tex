\documentclass{article}
\usepackage{amsmath, circuitikz}
\usepackage[margin=1in]{geometry}
\begin{document}

\section*{Kirchhoff's Law Problem}

\textbf{Problem:}

Consider the following circuit:

\begin{center}
\begin{circuitikz}[american]
    \draw
    (0,0) to[battery1, l=10V] (0,4)
          to[R, l=$R_1 = 2$] (4,4)
          to[R, l=$R_2 = 3$] (4,0)
          to[R, l=$R_3 = 5$] (0,0);
\end{circuitikz}
\end{center}

All components are in series in a single loop powered by a 10V battery. Use Kirchhoff’s Voltage Law to calculate the current \( I \) flowing in the circuit and the voltage drops across each resistor.

\vspace{1em}
\textbf{Solution:}

Using Kirchhoff’s Voltage Law (KVL), we sum the potential differences around the closed loop:

\[
V - V_{1} - V_{2} - V_{3} = 0
\]

Ohm's Law gives:

\[
V_{1} = I R_1
\]
\[
 V_{2} = I R_2
\]
\[
V_{3} = I R_3
\]

Substitute:

\[
10 - 2I - 3I - 5I = 0
\]

\[
10 - 10I = 0 
\]

\[
I = 1
\]

Now compute voltage drops:

\[
V_{1} = 1 \times 2
\]
\[
V_{2} = 1 \times 3
\]
\[
V_{3} = 1 \times 5
\]

\textbf{Final Answer:}

\begin{itemize}
  \item Current in the circuit: \( \boxed{1\,\text{A}} \)
  \item Voltage drop across \( R_1 \): \( \boxed{2\,\text{V}} \)
  \item Voltage drop across \( R_2 \): \( \boxed{3\,\text{V}} \)
  \item Voltage drop across \( R_3 \): \( \boxed{5\,\text{V}} \)
\end{itemize}

\end{document}
